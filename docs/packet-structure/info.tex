\documentclass[11pt,letterpaper]{article}

\usepackage[margin=1in]{geometry}
\usepackage{bytefield}

\begin{document}

\begin{figure}[h!]
  \centering
  \begin{bytefield}[bitwidth=1.1em]{32}
    \bitheader{0-31} \\
    \wordbox{1}{Preamble} \\
    \begin{rightwordgroup}{Header}
      \wordbox{1}{Source IP Address} \\ % ip addr that sent to transmitter

      \wordbox{1}{Transmitter IP Address} \\ % transmitter ip addr

      & \bitbox{8}{SN} % sequence number of the packet (x of 30)
      & \bitbox{16}{Data Length} % number of bits of data (max 4096)
      & \bitbox{8}{Reserved} \\

      \wordbox{4}{MD5 Checksum}
    \end{rightwordgroup} \\
    \wordbox{5}{Data}
  \end{bytefield}
\end{figure}

The packet consists of these fields:
\begin{itemize}
\item \textbf{Preamble} - a series of alternating 1s and 0s
\item \textbf{Source IP Address} - the IP address of the machine that sent the data to the transmitter
\item \textbf{Transmitter IP Address} - the IP address of the machine transmitting the data
\item \textbf{SN} (8 bits) - the sequence number of the packet. Since each packet is transmitted 30 times, this is a number between 1 and 30
\item \textbf{Data Length} (16 bits) - the number of bytes of data contained in the packet
\item \textbf{Reserved} (8 bits) - reserved for future use
\item \textbf{MD5 Checksum} - the MD5 checksum of the data calculated prior to transmission
\item \textbf{Data} - the data payload of the packet; at most 4096 bits
\end{itemize}

\end{document}
%%% Local Variables:
%%% mode: latex
%%% TeX-master: t
%%% End:
